\section{Discussão de Resultados}

\par Após a realização da experiência foi possível extrair diversas conclusões.

\par Relativamente à compressão adiabática, verificou-se que se obteve para a constante experimental o valor $\alpha=1.400\pm0.002$, possuíndo por conseguinte um desvio à exactidão de 0$\%$ e um desvio à precisão de $0.143%$. Isto implica, naturalmente, que o processo de compressão levado a cabo foi perfeitamente adiabático. De facto, verificou-se que o calor libertado pelo sistema possui o valor de $Q=(0.20\pm0.27)J$, podendo ser congruentemente zero dentro da margem de incerteza experimental permitida, tal como esperado. Além disto, por integração da curva P(V) obteve-se um valor para o trabalho de W=$(16.54\pm02)$J, enquanto que a variação da energia interna do sistema se obteve como sendo $\DeltaU=(16.34\pm0.07)$J. Assim, verificamos que a diferença entre o trabalho e a variação da energia interna pode ser 0 dentro da margem de incerteza permitida pelos respectivos erros, corroborando a hipótese de uma compressão adiabática perfeita implicada pelo valor de $\alpha$ obtido. Como tal, podemos concluir que a aproximação teórica efectuada do ar atmosférico enquanto gás perfeito se revelou perfeitamente adequada para este caso.

\par Subsequentemente procedeu-se ao estudo de uma compressão isotérmica. Sabemos que para este processo, o valor de $\alpha$ teoricamente previsto corresponde a 1. Obteve-se experimentalmente para este um valor de $\alpha=1.015\pm0.001$, existindo então um desvio à exactidão de 1,5$\%$ e um desvio à precisão de $0.1\%$. Seria expectável, mediante a análise dos valores experimentais da temperatura, que tal ocorresse, visto a variação da temperatura do início para o fim da experiência ser de $\DeltaT=(0.9\pm0.2)K$, e por conseguinte não nula. Logo, o valor de alpha ligeiramente superior a 1 seria expectável, e coaduna-se com as previsões teóricas efectuadas resultantes da não perfeita isotermia das condições de compressão. De facto, verificou-se que a esta variação de temperatura estava associado um valor de variação de energia interna $\DeltaU=(0.19\pm0.04)$J, com um valor de calor libertado pelo sistema de $Q=-(12,40\pm0.24)J$ e um valor de trabalho de $(W=12.59\pm0.2)J$. Assim, notamos que a pequena variação de temperatura do sistema, fruto de um processo não inteiramente isotérmico, resultou numa variação de energia interna do sistema, o qual resulta logicamente de um trabalho superior ao calor libertado - o processo de compressão não é executado de forma inteiramente fluida e lenta, levando aos resultados obtidos. Todavia, como o processo isotérmico, apesar de lento, não é obviamente tão demorado quanto a permitir esta aproximação, resulta um $alpha$ diferente de 1.

\par Após efectuadas as duas compressões acima descritas, procedeu-se à execução da expansão homónima correspondente, isto é, uma expansão adiabática e uma expansão isotérmica. No tocante à expansão adiabática, obteve-se como valor experimental o valor $\alpha=1.405\pm0.005$, possuíndo por conseguinte um desvio à exactidão de 0.36$\%$ e um desvio à precisão também de $0.36%$. Notamos então que o processo ainda pode eventualmente ser perfeitamente adiabático dentro da margem de erro. De facto, verifica-se simultaneamente que $\DeltaU=(-5.553\pm0.007)$J, $Q=(-0.20\pm0.21)$J e $W=(5.349\pm0.2)$J. Assim, verifica-se que o calor libertado pode ser considerado nulo dentro da margem de incerteza, corroborando a hipótese de este ser um processo adiabático. Notamos então que não existe uma discrepância entre a situação de expansão e de compressão adiabática no tocante à qualidade da medição, ambas evidenciando o comportamento antecipado. 

\par É de notar todavia a existência de um dado fenómeno peculiar. Aquando da rápida expansão adiabática levada a cabo, a água presente no ar solidifique momentaneamente,podendo observar-se cristais de gelo, regressando todavia ao seu estado gasoso anterior quase instantaneamente. Tal é possibilitado pela súbita descida de temperatura do sistema aquando da expansão adiabática $\DeltaT=-(74,6\pm0.2)K$. De facto, constata-se que a temperatura final do sistema é inferior à temperatura de solidificação da água, corroborando o raciocínio levado a cabo para descrever o fenómeno observado.

\par Por fim, foi considerado o caso de uma expansão isotérmica. Para esta, obteve-se um valor de alpha de $\alpha=1.049\pm0.011$, possuindo por conseguinte um desvio à exactidão de 4,9$\%$ e um desvio à precisão de $1.04\%$, indiciando por conseguinte ser o pior método até agora utilizado em termos de qualidade embora esta não seja de modo algum má. De facto, este processo não pode ser considerado perfeitamente isotérmico, tendo-se constatado uma variação de temperaturas de valor $\DeltaT=-(6.6\pm0.2)K$. Tal resulta de ser mais difícil de executar um processo lento de expansão do que de compressão devido a ser mais difícil executar um processo fluido de levantamento do pistão, evidenciado pela maior dispersão de pontos experimentais evidenciada no gráfico pertinente a esta parte da experiência. De facto, constatou-se juntamente com isto que $\DeltaU=(-0.52\pm0.02)$J, $Q=(-3.2\pm0.2)$J e $W=(2.7\pm0.2)$J. Assim, podemos verificar que existe variação da energia interna fruto da variação da temperatura. 

\par É possível portanto constatar, de imediato, que as compressões apresentam uma maior qualidade de medição do que as expansões correspondentes no tocante à determinação dos valores experimentais de $\alpha$. Tal resulta de ser mais fácil de comprimir o pistão de forma cuidada do que levantá-lo aquando da expansão. Além disto, notamos que embora os processos adiabáticos pudessem ser considerados ambos de acordo com o modelo teórico, os processos isotérmicos não ocorrem de facto de modo isotérmico, visto existir uma diferença de temperatura em ambos os casos e, consequentemente, uma variação de energia interna que não deveria existir num processo isotérmico resultante de esta ser uma função dependente apenas da temperatura no modelo de um gás ideal. Assim, podemos concluir que em termos de execução, os erros resultam maioritariamente da dificuldade de manipular o pistão de uma forma fluida e contínua e do facto de o software apresentar um limite de tempo no qual se podem extrair dados o que limita o tempo que se pode demorar para efectuar os processos isotérmicos - o que, juntamente com o atrito estático na compressão e expansão aquando da manipulação do pistão impede a situação de equilíbrio térmico para cada instante.

\par Revela-se ainda necessário mencionar que no início e no final da obtenção de dados se calcularam o número de moles de ar presentes no interior do cilindro. Supostamente este deveria manter-se constante, fruto do facto de as torneiras se encontrarem totalmente fechadas. Todavia, nota-se que no caso das compressões existe um aumento do número de moles e nas expansões uma diminuição. Isto seria aparentemente contra-intuitivo. Todavia, facilmente se compreende o que ocorre quando temos em conta que existe um atraso do sensor em relação à medição. Assim, no fim da experiência, os valores da temperatura medidos correspondem a momentos anteriores - no caso da compressão o valor da temperatura registado será inferior ao real, por exemplo. Como PV=nRT, tem-se que uma menor temperatura implica um maior número de moles, o que explica o verificado, sendo que para a expansão o fenómeno contrário se verifica. Realçamos que no caso dos processos isotérmicos, por a variação de temperatura ser muito menor e consequentemente para o tempo de atraso do sensor ser quase desprezável, verifica-se que o número de moles se mantém quase invariável.

\section{Conclusão}
\par Para o caso adiabático obteve-se para $\alpha$ os valores $\alpha=1.400\pm0.002$ e $\alpha=1.405\pm0.005$ no caso da compressão e expansão, respectivamente. Tendo em conta os valores para o trabalho e o calor envolvidos em cada processo podemos concluir que foi possível efectuar processos perfeitamente adiabáticos, indiciando a qualidade da aproximação do modelo dos gases perfeitos. Por outro lado, no referente aos processos isotérmicos não foi possível garantir as condições de isotermia, ocorrendo uma variação de temperatura quer na compressão (ainda que inferior a 1K) quer na expansão (6.6K). A expansão, em geral, revelou-se mais difícil de efectuar devido a dificuldades no manuseamento do material. Por fim, foi possível observar o fenómeno de formação de cristais de gelo aquando do processo de expansão adiabática. Notamos que esta experiência encontra-se sempre limitada pela dificuldade de manipular de forma fluida o pistão bem como do atraso dos sensores, que induzem erros impossíveis de evitar.